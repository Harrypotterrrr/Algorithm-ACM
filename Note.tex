\documentclass{article}
\usepackage{xeCJK} % 允许斜体和粗体
\usepackage{amsmath}
\usepackage{amssymb}
\usepackage{graphicx}
\usepackage{geometry}
\usepackage{bm}
\geometry{a4paper,scale=0.8}
\begin{document}\LARGE
\title{Script & Note}
    \section{\Huge Math}
    \subsection{\LARGE POJ 2115}
    According to the probelm description, we are looking for an $x$, such that
    \[A + Cx \equiv B (mod\ 2^k)\]
    then we can get
    \[Cx \equiv B-A (mod\ 2^k)\]
    from the Congruence equation quality, we will transfrom it into an equation
    \[2^kx + Cy= B-A\]
    is available, where $x$ and $y$ is what we are looking for.
    Now we turn to find the answer of
    \[2^kx_0 + Cy_0= gcd(2^k,C)\]
    that is completely the Euclid form.
    We will use Extended Euclid Algorithm to solve this function.
    As we get the answer of $y0$ by recurring, we will try to get $y$ through
    \[y=y_0*\frac{B-A}{gcd(a,b)}\]
    but for this problem, we must find the least positive one.
    Let $x_1,y_1$ satisfy:
    \[2^kx_1 + Cy_1= gcd(2^k,C)\]
    Observing these two equations, we can get:
    \[2^k(x_0-x_1)=C(y_1-y_0)\]
    and divided by $gcd(2^k,C)$ both side:
    \[\frac{2^k(x_0-x_1)}{gcd(2^k,C)}=\frac{C(y_1-y_0)}{gcd(2^k,C)}\]
    where $x_0-x_1$ is a multiple of $\frac{C}{gcd(2^k,C)}$, correspondently $y_1-y_0$ is a multiple of $\frac{2^k}{gcd(2^k,C)}$, for $2^k$ is prime to $gcd(2^k,C)$, and same to $C$.
    Thus we can conclude an significant equation :\\
    if we have $ax+by=c:$
    \[
    \left\{
    \begin{aligned}
        x &= x' + k\frac{b}{gcd(a,b)}\\
        y &= y' - k\frac{a}{gcd(a,b)}
    \end{aligned}
    \right.
    ,k\in Z
    \]
    so that we can find \textbf{the least positive one} matched to the problem, \textbf{(Noted that the equation above is irrelevent to the right constant!)}
    \[
    \left\{
    \begin{aligned}
    x=(x\%\frac{b}{gcd(a,b)} + \frac{b}{gcd(a,b)})\%\frac{b}{gcd(a,b)}\\
    y=(y\%\frac{a}{gcd(a,b)} + \frac{a}{gcd(a,b)})\%\frac{a}{gcd(a,b)}
    \end{aligned}
    \right.
    \]

    such we call \textbf{the least positive equation}\\
    And then we will use \textbf{the least positive equation} to find the answer of the problem finally.
    Here we can prove the Extended Euclid Algorithm:
    let
    \[
        \left\{
        \begin{aligned}
            &ax_1+ by_1= gcd(a,b)\\
            &bx_2+(a\%b)y_2=gcd(b,a\%b)
        \end{aligned}
        \right.
        ,a>b>0
        \]
        According to the Euclid pricipal, $gcd(a,b)=gcd(b,a\%b).$ we can get:
        \begin{align*}
        ax_1+by_1&=bx_2+(a\%b)y_2\\
        &=bx_2+(a-a/b*b)y_2\\
        &=ay_2+bx_2-a/b*by_2\\
        &=ay_2+b(x_2-a/b*y_2)
        \end{align*}

        relatively, 
        \[
        \left\{
        \begin{aligned}
            x_1 &= y_2\\
            y_1 &= x_2-a/b*y_2
        \end{aligned}
        \right.
        \]
        which is the recursion we need in recurring function of Extended Euclid algorithm.

    \subsection{\LARGE POJ 2115}
    we are to calculate 
    \begin{equation*}
    sum = \sum_{i=1}^Ngcd(N,i)
    \end{equation*}
    take N=6 as an example:
    \[
        \begin{tabular}{ccccccc}
            i&1&2&3&4&5&6\\
            gcd(i,6)&1&2&3&2&1&6\\
        \end{tabular}
    \]
    we take gcd(i,N) as $g_i$
    we can know $g_i$ appears exactly $\phi(\frac{N}{g_i})$ times, where $\phi$ is Euler function.
    because of:
    \[
    (\frac{N}{g_i},\frac{i}{g_i}) = 1  
    \]
    and the meaning of $\phi(x)$ is the number of figure prime to the $x$.\\
    Thus $\phi\frac{N}{g_i})$ corresponds to the number of $\frac{i}{g_i}$, which is also the number of $i$, such that $gcd(N,i)=g_i$\\ 
    On the other hand, if we take $i$, which is the factor of $N$, then $gcd(N,i)$ is exactly i, and the number of which we can calculated is $\phi(\frac{N}{g_i} = \frac{N}{i})$\\
    Therefore, traversing i range from 1 to N to find the factor of N is the essential optimization to the algorithm.

\end{document}